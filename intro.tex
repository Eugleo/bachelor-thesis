\chapwithtoc{Introduction}

Proteins are amino acid biopolymers that take part in most natural processes in living organisms. Among other things, they are vital for cell growth, reproduction, metabolism, and movement~\cite{johnson1994sequential, prescott1968regulation, ketelaar2004actin, elston1998energy}.

Protein function is highly dependent on its structure~\cite{orengo1999protein} which is at least partially determined by the sequence of the protein~\cite{anfinsen1973principles}. The structure is often stabilized by disulphide bonds~\cite{wedemeyer2000disulfide, mcauley2008contributions}, which can by extension directly or indirectly influence protein function~\cite{nagahara2011intermolecular}, and its regulation~\cite{chiu2019allosteric}. Thus, the study of disulphide bond positions is a key part of protein structure and function research. Furthermore, the knowledge of the positions of fixed-length cross-links --- such as the disulphide bonds --- can be utilized to constrain protein folding methods based on molecular dynamic simulations, improving their accuracy and performance~\cite{brodie2017solving}.

A popular method of disulphide bond mapping --- the determination of disulphide bond positions --- is tandem mass spectrometry. In experiments utilizing tandem mass spectrometry for mapping disulphide bonds, the protein is broken down to short peptide fragments, the mass of which is then analysed. The resulting mass spectra are searched for masses specific to peptides connected by a disulphide bond. The matches are used to identify the disulphide bonds in the analysed protein.

There exist many computational methods that are able to automatically identify simple cross-linked peptides from mass spectra~\cite{lakbub2018recent, liu2014facilitating}. However, identification of disulphide bond configurations that are more complex than the peptide-bond-peptide arrangement has proven to be challenging. Examples of these complex configurations include multiple interconnected peptides, or a single peptide with one or more intra-peptide bonds. Researchers often have to resort to labour-intensive partial reduction methods when studying proteins with these types of disulphide bond crosslinking~\cite{wu1997novel, li2013disulfide}.

In this thesis we first review the popular proteomic mass spectrometry methods, focusing on those that are relevant for disulphide bond characterization. We also mention some challenges caused by a common sample preparation protocol.

In the second chapter, we propose a custom method for automatic disulphide bond characterization using differential analysis, implemented in our command-line program \emph{Dibby}. Dibby's aim is to provide a richer set of options for fragment search, leading to better disulphide bond identifications even in complex proteins. To achieve this flexibility, Dibby extends the fragment search space to include most of the theoretically probable peptide fragments, independently of their disulphide bond configurations. The algorithms make use of the divide-and-conquer~\cite{smith1985design} and branch-and-bound~\cite{boyd2007branch} techniques which assist the affective exploration of the vast fragment search space. To keep the false discovery rate under control, we additionally perform basic differential analysis on reduced samples that do not contain any disulphide bonds.

Finally, we evaluate Dibby on measured and in-silico generated datasets, and show that thanks to the flexibility of its matching algorithm, Dibby is able to identify even the more complicated peptides with intra-peptide bonds.
