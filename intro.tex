
\chapwithtoc{Introduction}

\begin{enumerate}
  \item Existují proteiny, jsou klíčové pro funkci organismu, obstarávají většinu procesů v něm. Funkce proteinu je závislá na jejich struktuře, a ta je závislá mimo jiné i na nekovalentních interakcích jednotlivých aminokyselin. Tyto interakce probíhají přes vodíkové nebo disulfidické můstky.
  \item Vědět, kde tyto můstky jsou, může pomoci molecular dynamic simulations pro omezení vyhledávacího prostoru, propř. určitě i jiným věcem.
  \item Metod určování pozic SS můstků existuje spousta. Jedna z nich využívá tandemovou hmotnostní spektrometrii v kombinaci s kapalinovou chromatografií. To vše na částečně alkylovaném proteinu, který je rozložený trypsinem.
  \item V této práci jsme zvolili podobný postup (to jest LC-MSMS na tryptických peptidech), ale přidali jsme k němu in-silico matchování (di)peptidů na naměřená spektra pomocí novel divide and conquer metody. Tato metoda využívá toho, že dipeptidy mají specifický fragmentační pattern a navíc mají i jinou prekurzorovou hmotu.
  \item Tuto metodu ověřujeme na několika naměřených proteinech, a máme svkělé výsledky (hopefully).
\end{enumerate}


% Introduction should answer the following questions, ideally in this order:
% \begin{enumerate}
% \item What is the nature of the problem the thesis is addressing?
% \item What is the common approach for solving that problem now?
% \item How this thesis approaches the problem?
% \item What are the results? Did something improve?
% \item What can the reader expect in the individual chapters of the thesis?
% \end{enumerate}

% Expected length of the introduction is between 1--4 pages. Longer introductions may require sub-sectioning with appropriate headings --- use \texttt{\textbackslash{}section*} to avoid numbering (with section names like `Motivation' and `Related work'), but try to avoid lengthy discussion of anything specific. Any ``real science'' (definitions, theorems, methods, data) should go into other chapters.
% \todo{You may notice that this paragraph briefly shows different ``types'' of `quotes' in TeX, and the usage difference between a hyphen (-), en-dash (--) and em-dash (---).}

% It is very advisable to skim through a book about scientific English writing before starting the thesis. I can recommend `\citetitle{glasman2010science}' by \citet{glasman2010science}.
