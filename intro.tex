
\chapwithtoc{Introduction}

The study of protein structure and function is key to many areas in the modern science. Disulfide bonds play an important role in stabilising the protein structure~\cite{wedemeyer2000disulfide}, and can directly or indirectly influence its function~\cite{nagahara2011intermolecular}. Lately, mass spectrometry has become a popular tool in disulfide bond identification and mapping in proteins with known sequence. While there exist computational methods that are able to automatically indentify simple crosslinked pairs of peptides in mass spectrometric data~\cite{lakbub2018recent, liu2014facilitating}, indentification of more complex clusters of interlinked peptides has proved to be challenging, and researches often have to resort to labour intensive partial reduction methods~\cite{wu1997novel, li2013disulfide}.

In this thesis we first review the popular proteomic mass spectrometry methods and experiment design choices, focusing on those that are relevant for disulfide bond characterisation. Then we mention some challenges in sample preparation and data analysis regarding the study of disulfie bond linkages. Finally we propose a new command-line program for differential disulfide bond characterisation called \emph{Dibby}. It makes use of the \emph{divide and conquer}~\cite{smith1985design} and \emph{branch and bound}~\cite{boyd2007branch} techniques to indentify even the more complicated peptides with closely-spaced interlinked cysteines, or nested disulfide bonds. Furthermore, data from reduced samples, without any disulfide bonds, are used to lower the false discovery rate. We evaluate the program on measured and in-silico generated datasets, concluding that it poses a valid research direction in computation disulfide bond mapping.