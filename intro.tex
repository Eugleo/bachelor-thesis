
\chapwithtoc{Introduction}

Proteins are amino acid biopolymers that take part in most natural processes in living organisms. Among other things, they are vital for cell growth, reproduction, metabolism, and movement~\cite{johnson1994sequential, prescott1968regulation, ketelaar2004actin, elston1998energy}. Proteins can also frequently serve as potential drug targets, because they play a crucial role in most diseases~\cite{konopatskaya2010protein, yamin2008amyloid}.

Protein function is highly dependent on its structure~\cite{orengo1999protein} which is at least partially determined by the sequence of the protein~\cite{anfinsen1973principles}. The structure is oftentimes stabilised by disulfide bonds~\cite{wedemeyer2000disulfide, mcauley2008contributions}, which can by extension directly or indirectly influence protein function~\cite{nagahara2011intermolecular}, and its regulation~\cite{chiu2019allosteric}. Thus the study of disulfide bond positions is a key part of protein structure and function research. Furthermore, the knowledge of the positions of fixed-length crosslinks --- such as disulfide bonds --- can be utilised to constrain protein folding methods based on molecular dynamic simulations, improving their accuracy and performace~\cite{brodie2017solving}.

A popular method of disulfide bond mapping --- the determination of disulfide bond positions --- is tandem mass spectrometry. In tandem mass spectrometry experiments aiming to map disulfide bonds, the studied protein is broken down to short peptide fragments, whose mass is analysed. The resulting mass spectra are searched for masses specific to peptides connected by a disulfide bond. The matches are used to identify the disulfide bonds that were present in the analysed protein.

There exist many computational methods that are able to automatically indentify simple crosslinked peptides from mass spectra~\cite{lakbub2018recent, liu2014facilitating}. However, indentification of disulfide bond configurations that are more complex than the peptide-bond-peptide arrangement has proved to be challenging. Examples of these complex disulfide bond configurations include multiple interconnected peptides, or a single peptide with one or more intrapeptide bonds. Researchers often have to resort to labour-intensive partial reduction methods when studying proteins with complicated patterns of disulfide bond crosslinking~\cite{wu1997novel, li2013disulfide}.

In this thesis we first review the popular proteomic mass spectrometry methods, focusing on those that are relevant for disulfide bond characterisation. We also mention some challenges caused by a common sample preparation protocol.

In the second chapter, we propose a new method for automatic disulfide bond characterisation using differential analysis, implemented in our command-line program \emph{Dibby}. Dibby's aim is to provide a richer set of options for fragment search, leading to better disulfide bond identifications even in complex proteins. To achieve this flexibility, Dibby extends the fragment search space to include most theoretically possible peptide fragments, independently of their disulfide bond configurations. The algorithms make use of the \emph{divide and conquer}~\cite{smith1985design} and \emph{branch and bound}~\cite{boyd2007branch} techniques to make the search more effective, even though the search space is very large. To keep the false discovery rate under control, we additionally perform simple differential analysis on reduced samples that do not contain any disulfide bonds.

Finally, we evaluate Dibby on measured and in-silico generated datasets, and show that thanks to the flexibility of its matching algorithm, Dibby is able to identify even the more complicated peptides with intrapeptide bonds.
