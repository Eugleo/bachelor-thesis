\chapter{Methods}
\label{chap:something}

Následující detaily doplní Martin.

\begin{enumerate}
	\item Trypsin máme z nějaké firmy, lysozym a BSA taky.
	\item Používáme orbitrapový analyzátor s přeností 10–15ppm. Pro filtraci a CID používáme neonový quadrupól.
	\item Naše metoda stojí na ručně psaném mutuálně rekurzivním biodegradabilním vegan-friendly algoritmu.
\end{enumerate}

\section{Algorithm}

\begin{enumerate}
	\item Algo má dvě části, obě fungují na principu divide and conquer a řeší problém podobný subset sum, ale s více sekvencemi.
	\item Je napsaný v Pythonu, ale viz kapitola Discussion, není to ideální volba.
\end{enumerate}

\subsection{Part one, precursor masses}

Tady výhledově bude podrobnější a formálnější popis subset sum, jeho řešení, a jak se můj problém a můj algo liší. Taky tady bude pseudokód.

\begin{enumerate}
	\item Vygenerujeme možné tryptické peptidy.
	\item Procházíme všechny prekurzorové hmoty a snažíme se z tryptických peptidů poskládat peptid s vhodnou hmotou.
	\item Toto ``skládání'' bere v úvahu jak missed cleavages, tak možné propojení peptidů SS můstky. Také uvažuje to, že cysteiny, které nejsou v můstku, jsou modifikovány alkylací (+57). Také uvažuje to, že některé (0–všechny) Met mohou být oxidovány (+16).
	\item Funguje to v podstatě jako divide and conquer rekurzivní algoritmus na subset sum.
	\item Výstupem je pro každé spektrum seznam tryptických peptidů, které je možné spojit do peptidu s danou prekurzorovou hmotou. U tohoto seznamu je také určeno, kolik je v něm Cys můstků (tj vlastně kolik je v něm alkylovaných modifikovaných Cys) a kolik je v něm modifikovaných Met.
	\item Běžně pro jedno spektrum získáme 0–3 takovéto peptidy.
	\item (hisotogram počtu namatchovaných prekurzorů)
	\item SLožitost algortmu je nějak šíleně exponenciální a ještě s velkými multiplikativními konstantami, jak tam je spousta těch maybe-modifikací, ale tím, že to je NP, tak se s tím holt musím smířit. Běží to zhruba minutu pro 13000 spekter a středně velký protein (LYS), což je pro praxi ok.
	\item Modifikace jsou konfigurovatelné., stejně jako hmota alkylace, a počet dovolených interpeptidových můstků.
\end{enumerate}

\subsubsection{Differences from subset sum}

\begin{enumerate}
	\item V sekvenci hledáme $n$ souvislých podsekvencí (v subset sum je to jakoby až $n$, protože tam žádná souvislost není v podmínce). $n$ určuje, kolik interpeptide SS dovolíme — pokud chceme jen dipeptidy, tak $n=1$, pokud i tripeptidy $n=2$ atp.
	\item Nedá se moc dělat dynamické programování, protože všechny mass jsou floaty. Sice nehledáme přesný sum a máme nějakou toleranci, její velikost ale závisí na současném stavu algoritmu (protože to je relativní chyba). Asi bychom to ale mohli ignorovat, všechny ty floaty vynásobit stem a jet normálně na přesný součet, viz diskuze.
	\item Větvení je vícero, protože můžeme skočit na další sekvenci, nebo přibrat modifikace současného rezidua atp.
	\item (obrázek větvení, srovnání subset sum a tohohle šílenství)
\end{enumerate}

\subsection{Part two, matching fragments}

\begin{enumerate}
	\item V zásadě to samé, jako v part 1, ale ještě o level kombinatoricky výbuchovatější.
	\item V každém spektru procházíme všechny naměřené hmoty fragmentů a snažíme se k nim in-silico vygenerovat fragment z nějakého z proteinů, které jsme k danému spektru vybrali v první části.
	\item Konkrétní řešení je zase rekurzivní a opět podobné subset sum — procházíme postupně peptid a rozhodujeme se, jestli uděláme break, nebo ne.
\end{enumerate}

Problémy jsou následující.

\begin{enumerate}
	\item V první části nevíme, kde přesně jsou můstky a modifikace, jen kolik jich je dohromady
	\item Nevíme, jakou charge měl fragment, který danou stopu vygeneroval, musíme tedy vyzkoušet generovat fragmenty pro všechny charge (1 až charge prekurzoru, tu známe).
	\item Vznikají nám neutral lossy, ale nemusí.
	\item Mohou se breakovat SS, ale nemusí. Pokud se breaknou, existují čtyři možnosti, co tam po SS můstku zbylo.
	\item Fragmenty mohou vznikat dvěma breaky — můžeme tedy mít kombinované b+y fragmenty.
	\item Opět (prakticky) nemůžeme použít dynamické programování.
	\item (obrázek nějakého složitého fragmentu?)
\end{enumerate}

Níže bude opět pseudokód a popis algoritmu podobbný tomu v části 1.

\section{Evaluation}

\begin{enumerate}
	\item (jak zpracováváme výstup algo?)
	\item (srovnáváme ho s něčím? jak?)
	\item (jaké je kritérium toho, že řekneme ``tady to je podezřelé, mrkni, jeslti tam není můstek'')
\end{enumerate}


% \section{Example with some mathematics}
% \label{sec:demo}

% \begin{defn}[Triplet]\label{defn:x}
% Given stuff $X$, $Y$ and $Z$, we will write a \emph{triplet} of the stuff as $(X,Y,Z)$.
% \end{defn}

% \newcommand{\Col}{\textsc{Colour}}

% \begin{thm}[Car coloring]\label{thm:y}
% All cars have the same color. More specifically, for any set of cars $C$, we have
% $$(\forall c_1, c_2 \in C)\:\Col(c_1) = \Col(c_2).$$
% \end{thm}

% \begin{proof}
% Use induction on sets of cars $C$. The statement holds trivially for $|C|\leq1$. For larger $C$, select 2 overlapping subsets of $C$ smaller than $|C|$ (thus same-colored). Overlapping cars need to have the same color as the cars outside the overlap, thus also the whole $C$ is same-colored.\todo{This is plain wrong though.}
% \end{proof}

% \begin{table}
% \centering
% {\footnotesize\sf
% \begin{tabular}{llrl}
% \toprule
% Column A & Column 2 & Numbers & More \\
% \midrule
% Asd & QWERTY & 123123 & -- \\
% Asd qsd 1sd & \textcolor{red}{BAD} & 234234234 & This line should be helpful. \\
% Asd & \textcolor{blue}{INTERESTING} & 123123123 & -- \\
% Asd qsd 1sd & \textcolor{violet!50}{PLAIN WEIRD} & 234234234 & -- \\
% Asd & QWERTY & 123123 & -- \\
% \addlinespace % a nice non-intrusive separator of data groups (or final table sums)
% Asd qsd 1sd & \textcolor{green!80!black}{GOOD} & 234234299 & -- \\
% Asd & NUMBER & \textbf{123123} & -- \\
% Asd qsd 1sd & DIFFERENT & 234234234 & (no data) \\
% \bottomrule
% \end{tabular}}
% \caption{An example table. Table caption should clearly explain how to interpret the data in the table. Use some visual guide, such as boldface or color coding, to highlight the most important results (e.g., comparison winners).}
% \label{tab:z}
% \end{table}

% \begin{figure}
% \centering
% \includegraphics[width=.6\linewidth]{img/ukazka-obr02.pdf}
% \caption{A figure with a plot, not entirely related to anything. If you copy the figures from anywhere, always refer to the original author, ideally by citation (if possible). In particular, this picture --- and many others, also a lot of surrounding code --- was taken from the example bachelor thesis of MFF, originally created by Martin Mareš and others.}
% \label{fig:g}
% \end{figure}

% \begin{figure}
% \centering
% \tikzstyle{box}=[rectangle,draw,rounded corners=0.5ex,fill=green!10]
% \begin{tikzpicture}[thick,font=\sf\scriptsize]
% \node[box,rotate=45] (a) {A test.};
% \node[] (b) at (4,0) {Node with no border!};
% \node[circle,draw,dashed,fill=yellow!20, text width=6em, align=center] (c) at (0,4) {Ugly yellow node.\\Is this the Sun?};
% \node[box, right=1cm of c] (d) {Math: $X=\sqrt{\frac{y}{z}}$};
% \draw[->](a) to (b);
% \draw[->](a) to[bend left=30] node[midway,sloped,anchor=north] {flow flows} (c);
% \draw[->>>,dotted](b) to[bend right=30] (d);
% \draw[ultra thick](c) to (d);

% \end{tikzpicture}
% \caption{An example diagram typeset with TikZ.}
% \label{fig:schema}

% % \begin{algorithm}
% % \begin{algorithmic}
% % \Function{ExecuteWithHighProbability}{$A$}
% % 	\State $r \gets$ a random number between $0$ and $1$
% % 	\State $\varepsilon \gets 0.0000000000000000000000000000000000000042$
% % 	\If{$r\geq\varepsilon$}
% % 		\State execute $A$ \Comment{We discard the return value}
% % 	\Else
% % 		\State print: \texttt{Not today, sorry.}
% % 	\EndIf
% % \EndFunction
% % \end{algorithmic}
% % \caption{Algorithm that executes an action with high probability. Do not care about formal semantics in the pseudocode --- semicolons, types, correct function call parameters and similar nonsense from `realistic' languages can be safely omitted. Instead make sure that the intuition behind (and perhaps some hints about its correctness or various corner cases) can be seen as easily as possible.}
% % \label{alg:w}
% % \end{algorithm}

% \section{Extra typesetting hints}

% Do not overuse text formatting for highlighting various more or less parts of your sentences; if an idea cannot be communicated without formatting, the sentence probably needs rewriting anyway.

% Most importantly, do \underline{not} overuse bold text, which is designed to literally \textbf{shine from the page} to be the first thing that catches the eye of the reader. More precisely, use bold text only for `navigation' elements that need to be seen first, such as headings, list item names, and figure numbers.

% Use underline only in dire necessity, such as in the previous paragraph where it was inevitable to ensure that the reader remembers to never typeset boldface text manually again.

% Use \emph{emphasis} to highlight the first occurrences of important terms that the reader should notice. The feeling the emphasis produces is, roughly, ``Oh my --- what a nicely slanted word! Surely I expect it be important for the rest of the thesis!''

% Finally, never draw a vertical line (e.g., in a table or around figures), ever. Vertical lines outside of the figures are ugly.


% \chapter{More complicated chapter}
% \label{chap:math}

% After the reader gained sufficient knowledge to understand your problem in \cref{chap:refs}, you can jump to your own advanced material and conclusions.

% You will need definitions (see \cref{defn:x} below in \cref{sec:demo}), theorems (\cref{thm:y}), general mathematics, algorithms (\cref{alg:w}), and tables (\cref{tab:z})\todo{See documentation of package \texttt{booktabs} for hints on typesetting tables. As a main rule, \emph{never} draw a vertical line.}. \Cref{fig:f,fig:g} show how to make a nice figure. See \cref{fig:schema} for an example of TikZ-based diagram. Cross-referencing helps a lot to keep the necessary parts of the narrative close --- use references to the previous chapter with theory wherever it seems that the reader could have forgotten the required context.

% \section{Example with some mathematics}
% \label{sec:demo}

% \begin{defn}[Triplet]\label{defn:x}
% Given stuff $X$, $Y$ and $Z$, we will write a \emph{triplet} of the stuff as $(X,Y,Z)$.
% \end{defn}

% \newcommand{\Col}{\textsc{Colour}}

% \begin{thm}[Car coloring]\label{thm:y}
% All cars have the same color. More specifically, for any set of cars $C$, we have
% $$(\forall c_1, c_2 \in C)\:\Col(c_1) = \Col(c_2).$$
% \end{thm}

% \begin{proof}
% Use induction on sets of cars $C$. The statement holds trivially for $|C|\leq1$. For larger $C$, select 2 overlapping subsets of $C$ smaller than $|C|$ (thus same-colored). Overlapping cars need to have the same color as the cars outside the overlap, thus also the whole $C$ is same-colored.\todo{This is plain wrong though.}
% \end{proof}

% \begin{table}
% % uncomment the following line if you use the fitted top captions for tables
% % (see the \floatsetup[table] comments in `macros.tex`.
% %\floatbox{table}[\FBwidth]{
% \centering\footnotesize\sf
% \begin{tabular}{llrl}
% \toprule
% Column A & Column 2 & Numbers & More \\
% \midrule
% Asd & QWERTY & 123123 & -- \\
% Asd qsd 1sd & \textcolor{red}{BAD} & 234234234 & This line should be helpful. \\
% Asd & \textcolor{blue}{INTERESTING} & 123123123 & -- \\
% Asd qsd 1sd & \textcolor{violet!50}{PLAIN WEIRD} & 234234234 & -- \\
% Asd & QWERTY & 123123 & -- \\
% \addlinespace % a nice non-intrusive separator of data groups (or final table sums)
% Asd qsd 1sd & \textcolor{green!80!black}{GOOD} & 234234299 & -- \\
% Asd & NUMBER & \textbf{123123} & -- \\
% Asd qsd 1sd & DIFFERENT & 234234234 & (no data) \\
% \bottomrule
% \end{tabular}
% %}{  % uncomment if you use the \floatbox (as above), erase otherwise
% \caption{An example table.  Table caption should clearly explain how to interpret the data in the table. Use some visual guide, such as boldface or color coding, to highlight the most important results (e.g., comparison winners).}
% %}  % uncomment if you use the \floatbox
% \label{tab:z}
% \end{table}

% \begin{figure}
% \centering
% \includegraphics[width=.6\linewidth]{img/ukazka-obr02.pdf}
% \caption{A figure with a plot, not entirely related to anything. If you copy the figures from anywhere, always refer to the original author, ideally by citation (if possible). In particular, this picture --- and many others, also a lot of surrounding code --- was taken from the example bachelor thesis of MFF, originally created by Martin Mareš and others.}
% \label{fig:g}
% \end{figure}

% \begin{figure}
% \centering
% \tikzstyle{box}=[rectangle,draw,rounded corners=0.5ex,fill=green!10]
% \begin{tikzpicture}[thick,font=\sf\scriptsize]
% \node[box,rotate=45] (a) {A test.};
% \node[] (b) at (4,0) {Node with no border!};
% \node[circle,draw,dashed,fill=yellow!20, text width=6em, align=center] (c) at (0,4) {Ugly yellow node.\\Is this the Sun?};
% \node[box, right=1cm of c] (d) {Math: $X=\sqrt{\frac{y}{z}}$};
% \draw[->](a) to (b);
% \draw[->](a) to[bend left=30] node[midway,sloped,anchor=north] {flow flows} (c);
% \draw[->>>,dotted](b) to[bend right=30] (d);
% \draw[ultra thick](c) to (d);

% \end{tikzpicture}
% \caption{An example diagram typeset with TikZ.}
% \label{fig:schema}
% \end{figure}

% \begin{algorithm}
% \begin{algorithmic}
% \Function{ExecuteWithHighProbability}{$A$}
% 	\State $r \gets$ a random number between $0$ and $1$
% 	\State $\varepsilon \gets 0.0000000000000000000000000000000000000042$
% 	\If{$r\geq\varepsilon$}
% 		\State execute $A$ \Comment{We discard the return value}
% 	\Else
% 		\State print: \texttt{Not today, sorry.}
% 	\EndIf
% \EndFunction
% \end{algorithmic}
% \caption{Algorithm that executes an action with high probability. Do not care about formal semantics in the pseudocode --- semicolons, types, correct function call parameters and similar nonsense from `realistic' languages can be safely omitted. Instead make sure that the intuition behind (and perhaps some hints about its correctness or various corner cases) can be seen as easily as possible.}
% \label{alg:w}
% \end{algorithm}

% \section{Extra typesetting hints}

% Do not overuse text formatting for highlighting various important parts of your sentences. If an idea cannot be communicated without formatting, the sentence probably needs rewriting anyway. Imagine the thesis being read aloud as a podcast --- the storytellers are generally unable to speak in boldface font.

% Most importantly, do \underline{not} overuse bold text, which is designed to literally \textbf{shine from the page} to be the first thing that catches the eye of the reader. More precisely, use bold text only for `navigation' elements that need to be seen and located first, such as headings, list item leads, and figure numbers.

% Use underline only in dire necessity, such as in the previous paragraph where it was inevitable to ensure that the reader remembers to never typeset boldface text manually again.

% Use \emph{emphasis} to highlight the first occurrences of important terms that the reader should notice. The feeling the emphasis produces is, roughly, ``Oh my --- what a nicely slanted word! Surely I expect it be important for the rest of the thesis!''

% Finally, never draw a vertical line, not even in a table or around figures, ever. Vertical lines outside of the figures are ugly.
