\chapter{Differential characterisation of disulfide bonds}\label{chap:methods}

\paragraph{Reagents and instrumentation} Trypsin was purchased from Roche, lysosyme and lipase were obtained from Sigma-Aldrich. The used liquid chromatography setup was RSLC nano Ultimate 3000 (Thermo Scientific), for tandem mass spectrometry Orbitrap Fusion Lumos (Thermo Scientific) was employed.

\paragraph{Obtaining the data} We have anaylzed two proteins, lysosyme (LYS) and lipase (LIP). Two samples of each have been obtained in separate sample preparation pathways; in the AT sample preparation pathway, the samples have been alkylated without prior reduction, while in the RAT pathway, reduction took place before the alkylation. In both cases the alkylation has been done by iodacetamide (IAA), resulting in a covalent addition of a carbamidomethyl group (57.07 Da) onto cysteines without DBs; the reduction was achieved by dithiothreitol (DTT). After that, all four samples have been fully digested by trypsine, resulting in two sets of tryptides for each of the two analysed proteins. The samples have been treated the same for the rest of the experiment. After digestion, the peptides had undergone reverse-LC separation that was directly connected to the mass spectrometer through a nanospray ion source. After the measurement of MS\textsuperscript{1} spectra on a quadrupole, the precursors continue on to be fragmented with HCD\@. The MS\textsuperscript{2} spectra were analysed on an orbitrap analyser, as is implied by the use of HCD fragmentation. Finally, raw MS\textsuperscript{2} match data was exported to mgf files.

The data have been kindly provided by a IOCB mass specrometry research group (J. Cvačka), and have been measured a few years ago without any direct connection to this thesis.

\section{Dibbi, a program for disulfide bond visualisation}

A Python program termed Dibbi has been developed to analyse the positions of DBs in a protein. The input to Dibbi is the sequence of the analysed protein, and the mgf file with the measured MS\textsuperscript{2} data. It then provides a visualisation of computed scores of each of the possible DBs, and a file with the in-silico generated matches to the individual fragments and precursors, allowing for further manual analysis of the visualised results. Although Dibbi is perfectly able to operate only with the AT sample data, for best results it is preferable to analyse their RAT counterpart first. This enables the user to manually tune the parameters of the subsequent AT analysis, allowing for lower number of false positive hits.

Dibbi first assigns dynamically generated \emph{precursors} to the measured precursor matches, and then assigns in-silico generated fragments from the assigned precursors to the individual peaks in the spectra. Both of the algorithms make use of the divide and conquer~\cite{smith1985design} and branch-and-bound~\cite{boyd2007branch} techniques. Finally, the precurors are scored according to the quality of their fragment matches, among other things, and their score-weighted contributions are added together to form a general overview of the disulfide bonds in the protein. The precursor matching, fragment matching, and scoring algorthms are described in their own sections below.

\subsection{Assigning precursor mass}

The protein is first completely digested into peptides that are further undigestable; the protease of our choosing was trypsin, so we call these basal peptides \emph{tryptides}. As mentioned in the first chapter, a protease can sometimes miss a cleavage point, resulting in a peptie chain that is made from two (or more) contiguous basal tryptides connected with peptide bonds --- we call a chain of one or more tryptides a \emph{segment}. Finally, due to crosslinking with disulfide bridges, a \emph{precursor} is made out of one or more of these segments, joined with interpeptide disulfide bonds.

There can be one or more DB connecting a segment to another segment, and there can also be intrapeptide DBs. Cysteines that do not partake in a DB are alkylated, meaning their mass is effectively higher by some configurable constant. In addition, some amino acid residues are sometimes modified, too, such as the methionine undergoing oxidation. The algorithm treats the latter as optional, or ``variable'' modifications.

From the point of view of this part of the algorithm, it matters not where the individual DBs are located, or which exact methionines are modified, because in individual variations are not distiguishable on the basis of precursor mass. Thus, the precursors in the output of the algorithm only include the information about which segments are present, how many DBs there are --- always at least the number of segments minus 1 --- and how many of each kind of variable modifications there are. This information is passed to the next stage of the program.

As for the inner workings of the algorithm, each measured precursor mass is saved into the global variable \(TARGET\), and then the \textsc{FindPrec} function is caled with increasingly higher and higher starting points; in this way, every possible precursor match for the target is found. The \textsc{FindPrec} function is implemented using the divide and conquer technique (see \Cref{alg:findprec}). The function keeps a list of \Var{Selected} segments, and a pointer to \Var{TRYPTIDES}, and in each interation, it tries to

\begin{enumerate}
	\item Combine the possible modifications and DBs in such a way that the \Var{Selected} segments with the modifications have the correct mass, using the \textsc{Combine} function (see \algref{alg:findprec}{alg:findprec:combinations}). \textsc{Combine} implements a divide-and-conquer algorithm for a modified subset sum problem, in which assignments that are within some error boundary of the target are also considered a valid solution.
	\item Elongate the ``current'' segment it is building by adding the current tryptide to it (see \algref{alg:findprec}{alg:findprec:elongate}, and also the whole \Cref{alg:elongate}).
	\item End the current segment (effectively simulating a protease cleavage just before the current tryptide) and begin a new one that begins on the current tryptide, or on some tryptide after it. See \algref{alg:findprec}{alg:findprec:end}, and the whole \Cref{alg:newseg}.
\end{enumerate}

As we can see, the pointer to tryptides only ever moves to the right. This is a simple method of symmetry breaking.

\begin{algorithm}
	\begin{algorithmic}
		\Function{FindPrec}{$I, \mathit{Selected}, \mathit{Mass}, \mathit{Segments}, \mathit{Cys}, \mathit{Open}$}
		\State $\mathit{solutions} \gets$ empty list

		\nb{There is no hanging disulfide bond from previous segments}
		\nb{So let us try to combine residue modifications to find a solution}
		\If{not $\mathit{Open}$}\label{alg:findprec:combinations}
		\State $mods \gets$ calculate possible modifications from $\mathit{Selected}$
		\State $alks \gets$ calculate alkylation count based on seen non-bonded $\mathit{Cys}$
		\State $combinations \gets$ \Call{Combine}{$mass, mods, alks$}
		\State return a list of precursors generated from $combinations$
		\EndIf

		\nb{There are no further solutions, our $\mathit{Mass}$ can not get low enough}
		\If{$\mathit{Mass}$ is too high, or $i$ is at the end}
		\State return empty list
		\EndIf

		\nb{End this segment, connect next one with a disulfide bond}
		\nb{That is, if we have the segments budget and a free cysteine}
		\If{not $\mathit{Open}$, and $\mathit{Segments} > 0$, and $\mathit{Cys} > 0$}\label{alg:findprec:end}
		\State $S \gets$ \Call{NewSegment}{$I, \mathit{Selected}, \mathit{Mass}, \mathit{Segments}, \mathit{Cys}, \mathit{Open}$}
		\State concatenate $\mathit{solutions}$ with the list $S$
		\EndIf

		\nb{Elongate the current segment by one tryptide}
		\State $S \gets$ \Call{Elongate}{$I, \mathit{Selected}, \mathit{Mass}, \mathit{Segments}, \mathit{Cys}, \mathit{Open}$}\label{alg:findprec:elongate}
		\State concatenate $\mathit{solutions}$ with the list $S$
		\State return the list $\mathit{solutions}$

		\EndFunction
	\end{algorithmic}
	\caption{The main part of the precursor matching algorithm, in which all the braching occurs.}\label{alg:findprec}
\end{algorithm}


\begin{algorithm}
	\begin{algorithmic}
		\Function{Elongate}{$I, Selected, \mathit{Mass}, \mathit{Segments}, \mathit{Cys}, \mathit{Open}$}
		\nb{Prolong the current segment by one tryptide}

		\State $\mathit{tryptide} \gets$ the $I$-th tryptide from the list $\mathit{TRYPTIDES}$
		\State $\mathit{mass'} \gets$ the mass of $\mathit{tryptide}$ added to $\mathit{Mass}$
		\State $\mathit{cys'}\gets$ the number of cys in $\mathit{tryptide}$ added to $\mathit{Cys}$
		\Decl{cys'}{lower \var{cys'} by one if \Var{Open} is true, using a cys to close the bond}
		\State $\mathit{open'} \gets$ False if this tryptide had any cysteines, otherwise $\mathit{Open}$

		\nb{Call the original function}
		\State $S \gets$ \Call{FindPrec}{$I + 1, \mathit{Selected}, \mathit{mass'}, \mathit{Segments}, cys', \mathit{open'}$}
		\State return the list $S$

		\EndFunction
	\end{algorithmic}
	\caption{Elongates the currently built precursor segment, adding the current tryptide to it.}\label{alg:elongate}
\end{algorithm}


\begin{algorithm}
	\begin{algorithmic}
		\Function{NewSegment}{$I, \mathit{Selected}, \mathit{Mass}, \mathit{Segments}, \mathit{Cys}, \mathit{Open}$}
		\State $sel' \gets$ the currently ending segment added to $\mathit{Selected}$
		\State $\mathit{mass'} \gets$ subtract mass of $\ce{H2}$ from $\mathit{Mass}$, due to the new DB
		\nb{Update the budget, because of the newly started segment}
		\State $\mathit{seg'} \gets \mathit{Segments} - 1$
		\nb{Begin the bond with one of our cysteines}
		\State $\mathit{cys'}\gets \mathit{Cys} - 1$
		\nb{The new bond is waiting to get ``closed'' by a cys in the next run}
		\State $\mathit{open'} \gets$ True

		\nb{Finally, start the new segment, try all possible starting points}
		\For{all possible beginings of the next segment from $I + 1$ onward}
		\State $i' \gets$ the next beginning
		\State $S \gets$ \Call{AddSegment}{$i', selected', \mathit{mass'}, segments' cys', \mathit{open'}$}
		\State return the list $S$
		\EndFor

		\EndFunction
	\end{algorithmic}
	\caption{Ends the currently built precursor segment and begins a new one, beginning with the current tryptide, or any tryptide coming after it.}\label{alg:newseg}
\end{algorithm}


\subsection{Assigning fragment mass}

\begin{enumerate}
	\item V zásadě to samé, jako v part 1, ale ještě o level kombinatoricky výbuchovatější.
	\item V každém spektru procházíme všechny naměřené hmoty fragmentů a snažíme se k nim in-silico vygenerovat fragment z nějakého z proteinů, které jsme k danému spektru vybrali v první části.
	\item Konkrétní řešení je zase rekurzivní a opět podobné subset sum — procházíme postupně peptid a rozhodujeme se, jestli uděláme break, nebo ne.
\end{enumerate}

Problémy jsou následující.

\begin{enumerate}
	\item V první části nevíme, kde přesně jsou můstky a modifikace, jen kolik jich je dohromady
	\item Nevíme, jakou charge měl fragment, který danou stopu vygeneroval, musíme tedy vyzkoušet generovat fragmenty pro všechny charge (1 až charge prekurzoru, tu známe).
	\item Vznikají nám neutral lossy, ale nemusí.
	\item Mohou se breakovat SS, ale nemusí. Pokud se breaknou, existují čtyři možnosti, co tam po SS můstku zbylo.
	\item Fragmenty mohou vznikat dvěma breaky — můžeme tedy mít kombinované b+y fragmenty.
	\item Opět (prakticky) nemůžeme použít dynamické programování.
	\item (obrázek nějakého složitého fragmentu?)
\end{enumerate}

Níže bude opět pseudokód a popis algoritmu podobbný tomu v části 1.

\section{Precursor scoring and result visualisation}

\begin{enumerate}
	\item (jak zpracováváme výstup algo?)
	\item (srovnáváme ho s něčím? jak?)
	\item (jaké je kritérium toho, že řekneme ``tady to je podezřelé, mrkni, jeslti tam není můstek'')
\end{enumerate}


% \section{Example with some mathematics}
% \label{sec:demo}

% \begin{defn}[Triplet]\label{defn:x}
% Given stuff $X$, $Y$ and $Z$, we will write a \emph{triplet} of the stuff as $(X,Y,Z)$.
% \end{defn}

% \newcommand{\Col}{\textsc{Colour}}

% \begin{thm}[Car coloring]\label{thm:y}
% All cars have the same color. More specifically, for any set of cars $C$, we have
% $$(\forall c_1, c_2 \in C)\:\Col(c_1) = \Col(c_2).$$
% \end{thm}

% \begin{proof}
% Use induction on sets of cars $C$. The statement holds trivially for $|C|\leq1$. For larger $C$, select 2 overlapping subsets of $C$ smaller than $|C|$ (thus same-colored). Overlapping cars need to have the same color as the cars outside the overlap, thus also the whole $C$ is same-colored.\todo{This is plain wrong though.}
% \end{proof}

% \begin{table}
% \centering
% {\footnotesize\sf
% \begin{tabular}{llrl}
% \toprule
% Column A & Column 2 & Numbers & More \\
% \midrule
% Asd & QWERTY & 123123 & -- \\
% Asd qsd 1sd & \textcolor{red}{BAD} & 234234234 & This line should be helpful. \\
% Asd & \textcolor{blue}{INTERESTING} & 123123123 & -- \\
% Asd qsd 1sd & \textcolor{violet!50}{PLAIN WEIRD} & 234234234 & -- \\
% Asd & QWERTY & 123123 & -- \\
% \addlinespace % a nice non-intrusive separator of data groups (or final table sums)
% Asd qsd 1sd & \textcolor{green!80!black}{GOOD} & 234234299 & -- \\
% Asd & NUMBER & \textbf{123123} & -- \\
% Asd qsd 1sd & DIFFERENT & 234234234 & (no data) \\
% \bottomrule
% \end{tabular}}
% \caption{An example table. Table caption should clearly explain how to interpret the data in the table. Use some visual guide, such as boldface or color coding, to highlight the most important results (e.g., comparison winners).}
% \label{tab:z}
% \end{table}

% \begin{figure}
% \centering
% \includegraphics[width=.6\linewidth]{img/ukazka-obr02.pdf}
% \caption{A figure with a plot, not entirely related to anything. If you copy the figures from anywhere, always refer to the original author, ideally by citation (if possible). In particular, this picture --- and many others, also a lot of surrounding code --- was taken from the example bachelor thesis of MFF, originally created by Martin Mareš and others.}
% \label{fig:g}
% \end{figure}

% \begin{figure}
% \centering
% \tikzstyle{box}=[rectangle,draw,rounded corners=0.5ex,fill=green!10]
% \begin{tikzpicture}[thick,font=\sf\scriptsize]
% \node[box,rotate=45] (a) {A test.};
% \node[] (b) at (4,0) {Node with no border!};
% \node[circle,draw,dashed,fill=yellow!20, text width=6em, align=center] (c) at (0,4) {Ugly yellow node.\\Is this the Sun?};
% \node[box, right=1cm of c] (d) {Math: $X=\sqrt{\frac{y}{z}}$};
% \draw[->](a) to (b);
% \draw[->](a) to[bend left=30] node[midway,sloped,anchor=north] {flow flows} (c);
% \draw[->>>,dotted](b) to[bend right=30] (d);
% \draw[ultra thick](c) to (d);

% \end{tikzpicture}
% \caption{An example diagram typeset with TikZ.}
% \label{fig:schema}

% % \begin{algorithm}
% % \begin{algorithmic}
% % \Function{ExecuteWithHighProbability}{$A$}
% % 	\State $r \gets$ a random number between $0$ and $1$
% % 	\State $\varepsilon \gets 0.0000000000000000000000000000000000000042$
% % 	\If{$r\geq\varepsilon$}
% % 		\State execute $A$ \Comment{We discard the return value}
% % 	\Else
% % 		\State print: \texttt{Not today, sorry.}
% % 	\EndIf
% % \EndFunction
% % \end{algorithmic}
% % \caption{Algorithm that executes an action with high probability. Do not care about formal semantics in the pseudocode --- semicolons, types, correct function call parameters and similar nonsense from `realistic' languages can be safely omitted. Instead make sure that the intuition behind (and perhaps some hints about its correctness or various corner cases) can be seen as easily as possible.}
% % \label{alg:w}
% % \end{algorithm}

% \section{Extra typesetting hints}

% Do not overuse text formatting for highlighting various more or less parts of your sentences; if an idea cannot be communicated without formatting, the sentence probably needs rewriting anyway.

% Most importantly, do \underline{not} overuse bold text, which is designed to literally \textbf{shine from the page} to be the first thing that catches the eye of the reader. More precisely, use bold text only for `navigation' elements that need to be seen first, such as headings, list item names, and figure numbers.

% Use underline only in dire necessity, such as in the previous paragraph where it was inevitable to ensure that the reader remembers to never typeset boldface text manually again.

% Use \emph{emphasis} to highlight the first occurrences of important terms that the reader should notice. The feeling the emphasis produces is, roughly, ``Oh my --- what a nicely slanted word! Surely I expect it be important for the rest of the thesis!''

% Finally, never draw a vertical line (e.g., in a table or around figures), ever. Vertical lines outside of the figures are ugly.


% \chapter{More complicated chapter}
% \label{chap:math}

% After the reader gained sufficient knowledge to understand your problem in \cref{chap:refs}, you can jump to your own advanced material and conclusions.

% You will need definitions (see \cref{defn:x} below in \cref{sec:demo}), theorems (\cref{thm:y}), general mathematics, algorithms (\cref{alg:w}), and tables (\cref{tab:z})\todo{See documentation of package \texttt{booktabs} for hints on typesetting tables. As a main rule, \emph{never} draw a vertical line.}. \Cref{fig:f,fig:g} show how to make a nice figure. See \cref{fig:schema} for an example of TikZ-based diagram. Cross-referencing helps a lot to keep the necessary parts of the narrative close --- use references to the previous chapter with theory wherever it seems that the reader could have forgotten the required context.

% \section{Example with some mathematics}
% \label{sec:demo}

% \begin{defn}[Triplet]\label{defn:x}
% Given stuff $X$, $Y$ and $Z$, we will write a \emph{triplet} of the stuff as $(X,Y,Z)$.
% \end{defn}

% \newcommand{\Col}{\textsc{Colour}}

% \begin{thm}[Car coloring]\label{thm:y}
% All cars have the same color. More specifically, for any set of cars $C$, we have
% $$(\forall c_1, c_2 \in C)\:\Col(c_1) = \Col(c_2).$$
% \end{thm}

% \begin{proof}
% Use induction on sets of cars $C$. The statement holds trivially for $|C|\leq1$. For larger $C$, select 2 overlapping subsets of $C$ smaller than $|C|$ (thus same-colored). Overlapping cars need to have the same color as the cars outside the overlap, thus also the whole $C$ is same-colored.\todo{This is plain wrong though.}
% \end{proof}

% \begin{table}
% % uncomment the following line if you use the fitted top captions for tables
% % (see the \floatsetup[table] comments in `macros.tex`.
% %\floatbox{table}[\FBwidth]{
% \centering\footnotesize\sf
% \begin{tabular}{llrl}
% \toprule
% Column A & Column 2 & Numbers & More \\
% \midrule
% Asd & QWERTY & 123123 & -- \\
% Asd qsd 1sd & \textcolor{red}{BAD} & 234234234 & This line should be helpful. \\
% Asd & \textcolor{blue}{INTERESTING} & 123123123 & -- \\
% Asd qsd 1sd & \textcolor{violet!50}{PLAIN WEIRD} & 234234234 & -- \\
% Asd & QWERTY & 123123 & -- \\
% \addlinespace % a nice non-intrusive separator of data groups (or final table sums)
% Asd qsd 1sd & \textcolor{green!80!black}{GOOD} & 234234299 & -- \\
% Asd & NUMBER & \textbf{123123} & -- \\
% Asd qsd 1sd & DIFFERENT & 234234234 & (no data) \\
% \bottomrule
% \end{tabular}
% %}{  % uncomment if you use the \floatbox (as above), erase otherwise
% \caption{An example table.  Table caption should clearly explain how to interpret the data in the table. Use some visual guide, such as boldface or color coding, to highlight the most important results (e.g., comparison winners).}
% %}  % uncomment if you use the \floatbox
% \label{tab:z}
% \end{table}

% \begin{figure}
% \centering
% \includegraphics[width=.6\linewidth]{img/ukazka-obr02.pdf}
% \caption{A figure with a plot, not entirely related to anything. If you copy the figures from anywhere, always refer to the original author, ideally by citation (if possible). In particular, this picture --- and many others, also a lot of surrounding code --- was taken from the example bachelor thesis of MFF, originally created by Martin Mareš and others.}
% \label{fig:g}
% \end{figure}

% \begin{figure}
% \centering
% \tikzstyle{box}=[rectangle,draw,rounded corners=0.5ex,fill=green!10]
% \begin{tikzpicture}[thick,font=\sf\scriptsize]
% \node[box,rotate=45] (a) {A test.};
% \node[] (b) at (4,0) {Node with no border!};
% \node[circle,draw,dashed,fill=yellow!20, text width=6em, align=center] (c) at (0,4) {Ugly yellow node.\\Is this the Sun?};
% \node[box, right=1cm of c] (d) {Math: $X=\sqrt{\frac{y}{z}}$};
% \draw[->](a) to (b);
% \draw[->](a) to[bend left=30] node[midway,sloped,anchor=north] {flow flows} (c);
% \draw[->>>,dotted](b) to[bend right=30] (d);
% \draw[ultra thick](c) to (d);

% \end{tikzpicture}
% \caption{An example diagram typeset with TikZ.}
% \label{fig:schema}
% \end{figure}

% \begin{algorithm}
% \begin{algorithmic}
% \Function{ExecuteWithHighProbability}{$A$}
% 	\State $r \gets$ a random number between $0$ and $1$
% 	\State $\varepsilon \gets 0.0000000000000000000000000000000000000042$
% 	\If{$r\geq\varepsilon$}
% 		\State execute $A$ \Comment{We discard the return value}
% 	\Else
% 		\State print: \texttt{Not today, sorry.}
% 	\EndIf
% \EndFunction
% \end{algorithmic}
% \caption{Algorithm that executes an action with high probability. Do not care about formal semantics in the pseudocode --- semicolons, types, correct function call parameters and similar nonsense from `realistic' languages can be safely omitted. Instead make sure that the intuition behind (and perhaps some hints about its correctness or various corner cases) can be seen as easily as possible.}
% \label{alg:w}
% \end{algorithm}

% \section{Extra typesetting hints}

% Do not overuse text formatting for highlighting various important parts of your sentences. If an idea cannot be communicated without formatting, the sentence probably needs rewriting anyway. Imagine the thesis being read aloud as a podcast --- the storytellers are generally unable to speak in boldface font.

% Most importantly, do \underline{not} overuse bold text, which is designed to literally \textbf{shine from the page} to be the first thing that catches the eye of the reader. More precisely, use bold text only for `navigation' elements that need to be seen and located first, such as headings, list item leads, and figure numbers.

% Use underline only in dire necessity, such as in the previous paragraph where it was inevitable to ensure that the reader remembers to never typeset boldface text manually again.

% Use \emph{emphasis} to highlight the first occurrences of important terms that the reader should notice. The feeling the emphasis produces is, roughly, ``Oh my --- what a nicely slanted word! Surely I expect it be important for the rest of the thesis!''

% Finally, never draw a vertical line, not even in a table or around figures, ever. Vertical lines outside of the figures are ugly.
