%%% Choose a language %%%

\newif\ifEN
\ENtrue   % uncomment this for english
%\ENfalse   % uncomment this for czech

%%% Configuration of the title page %%%

% \def\ThesisTitleStyle{mff} % MFF style
%\def\ThesisTitleStyle{cuni} % uncomment for old-style with cuni.cz logo
\def\ThesisTitleStyle{natur} % uncomment for nature faculty logo

% \def\UKFaculty{Faculty of Mathematics and Physics}
\def\UKFaculty{Faculty of Science}

\def\UKName{Charles University in Prague} % this is not used in the "mff" style

% Thesis type names, as used in several places in the title
\def\ThesisTypeTitle{\ifEN BACHELOR THESIS \else BAKALÁŘSKÁ PRÁCE \fi}
\def\ThesisGenitive{\ifEN bachelor \else bakalářské \fi}
\def\ThesisAccusative{\ifEN bachelor \else bakalářskou \fi}


%%% Fill in your details %%%

\def\ThesisTitle{Differential discovery of protein features using tandem mass spectrometry}
\def\ThesisAuthor{Evžen Wybitul}
\def\YearSubmitted{2021}

% department assigned to the thesis
\def\Department{Department of Cell Biology}
% Is it a department (katedra), or an institute (ústav)?
\def\DeptType{Department}

\def\Supervisor{Miroslav Kratochvíl}
\def\SupervisorsDepartment{Luxembourg Centre for Systems Biomedicine}

% Study programme and specialization
\def\StudyProgramme{Bioinformatics}
\def\StudyBranch{BBINF}

\def\Dedication{%
  Dedication. First and foremost, I thank my supervisor and former colleague Miroslav Kratochvíl for his guidance and assistance with my research and with the writing of this thesis.

  I thank Josef Cvačka for providing the testing data. Martin Hubálek has my sincerest thanks for relentlessly answering my many questions related to mass spectrometry.

  My dear Lucka deserves special gratitude for enduring the many unsolicited mini-lectures about fragmentation in mass spectrometry, and for always being there for me when I needed it.

  Last but not least, I am grateful to my parents for their perpetual support throughout my studies at the university, and for the many compromises they had to make in order for me to successfully finish writing this thesis.
}

\def\AbstractEN{%
  Disulphide bonds are crucial to correct protein folding, and heavily influence protein function. Tandem mass spectrometry protein analysis is often used for the determination of disulphide bond positions, in combination with manual or computational interpretation methods. In this thesis we devise a program for automatic disulphide bond characterization called Dibby. Dibby identifies protein fragments in the fragmentation spectra, and uses the identified fragments to determine which cysteines were connected in the protein. The identification algorithm is able to identify even complex fragments with multiple disulphide bonds that are often missed by other methods. To reduce the fragment search space, we employ divide and conquer and branch and bound techniques. We evaluate Dibby on both measured and in-silico generated datasets, and find that it correctly identifies large portion of the present disulphide bonds with minimal manual interventions.
}

\def\AbstractCS{%
  Disulfidické můstky hrají důležitou roli při skládání proteinů a mají velký vliv na jejich funkci. K určování polohy disulfidických můstků v proteinech se často používá tandemová hmotnostní spektrometrie v kombinaci s manuální nebo výpočetní interpretací výsledků. V této práci představujeme progam Dibby, který má za cíl charakterizovat disulfidické můstky v proteinu. Dibby identifikuje proteinové fragmenty ve fragmentačních spektrech a s jejich pomocí určuje, kde v proteinu se disulfidické můstky nacházejí. Použitý identifikační algoritmus zvládá identifikovat i komplexní fragmenty s několika disulfidickými můstky, kterých si jiné metody často nevšimnou. Abychom při identifikaci fragmentů zmenšili vyhledávací prostor, využíváme metodu rozděl a panuj a metodu větví a mezí. Pomocí evaluace na naměřených i na uměle vygenerovaných datasetech jsme ověřili, že Dibby jen s minimálními manuálními zásahy správně rozpozná velkou část přítomných disulfidických můstků.
}

% 3 to 5 keywords (recommended), each enclosed in curly braces.
% Keywords are useful for indexing and searching for the theses by topic.
\def\Keywords{%
  {protein}, {mass spectrometry}, {disulphide bonds}, {disulphide bond mapping}, {fragmentation spectra labelling}
}

% If your abstracts are long and do not fit in the infopage, you can make the
% fonts a bit smaller by this setting. (Also, you should try to compress your abstract more.)
% Alternatively, consider increasing the size of the page by uncommenting the
% geometry modification in thesis.tex.
\def\InfoPageFont{}
%\def\InfoPageFont{\small}  %uncomment to decrease font size