\chapwithtoc{Conclusion}

In this thesis we have reviewed the current approaches to disulphide bond mapping, and the different trade-offs stemming from various choices regarding instrumentation, data preparation, and experiment design. We presented our own attempt at automatic computational disulphide bond characterization, a program toolkit aiming to identify even complicated clusters of disulphide bonds, including intra-peptide disulphide bonds. The source code of the program is available on GitHub\footnote{\url{https://github.com/Eugleo/dibby}}, along with the R scripts that were used for the subsequent analysis.

During the evaluation of the algorithm on real-world data we may have observed the occurrence of disulphide bond (DB) scrambling caused by tryptic protein digestion. Furthermore, due to the high number of false positives among the assigned fragments, some manual work is still needed to interpret the results. Another consequence of the high number of false positive matches is that the scoring metric needs to be quite complicated and opaque in order to weed them out. Some way of advanced utilization of data from the RAT samples may be needed to simplify the scoring metric.

A part of the simplification of the scoring metric will probably be a better utilization of the data from RAT samples; further research is needed on this front.

Despite these limitations and the challenging conditions regarding the data we used, we have successfully demonstrated Dibby's utility by identifying \glspl*{db} in lysozyme, including one intra-peptide \gls*{db}, and in our in-silico generated control dataset. We conclude that we have reached our goal of devising a general and flexible algorithm that is able to identify complicated cross-linked peptide fragments. We consider Dibby to be only a proof-of-concept at this stage of the development, but we believe that extensive matching of complex peptide fragments could serve as a good approach to \gls*{db} mapping.

\section*{Future work}

It would be interesting to see how well the algorithm works with data that are not digested with trypsin. This could confirm whether the high number of unexpected identifications had indeed been due to the issue of \gls*{db} scrambling during tryptic digestion. On this front, many other proteases would be viable, for example pepsin, or thermolysin~\cite{sung2016evaluation}. After a change in the sample preparation protocol, we expect the number of high-scoring ``bad'' variants to decrease, resulting in a stronger signal for the correctly identified \glspl*{db}, and in a reduced number of false positives. The same outcome could also be achieved by optimizing the scoring metric, and by more resourceful utilization of the data from RAT samples. Last but not least, data from other fragmentation sources could be used to make the dataset richer.

Further work could be done to optimize Dibby's performance, for example by using dynamic programming, a technique that is often used in divide-and-conquer algorithms. However, the employment of dynamic programming will be complicated by the multitudes of information that the algorithm has to keep track of --- such as neutral losses, amino acid modifications, cysteine alkylations, bond cleavages, error boundaries. Nonetheless, should the program be deployed in real-life scenarios in the future, reductions in compute time would be needed. Additional speed-ups could be achieved by implementing the program in a language more performant than Python, such as Julia.

Finally, due to its nature, Dibby has very high sensitivity; so high in fact that it is almost detrimental due to the quantity of false positives, were it not for the elaborate scoring and weighting post-processing steps. However, Dibby could be used to automatically label and analyse the various types of complex ions that show up in the fragmentation spectra, such as internal ions comprising multiple cross-linked peptides. If the sample were prepared very carefully, and the positions of \glspl*{db} in the protein were known, this knowledge could be used to manually discard most of the false positives. In this way, Dibby could be used to research and quantify the dissociation pathways of cross-linked peptides, potentially leading to more informed approaches to \gls*{db} mapping in the future.